\documentclass[12pt]{article}
\usepackage[a4paper]{geometry}
\usepackage{amsmath,amssymb,amsthm}
\theoremstyle{plain}
\begin{document}

\title{Solution of the Legible Math Problem}
\author{banteg}
\date{April 1, 2025}
\maketitle

\section*{System of Equations}
\begin{align*}
 z\,e\,r\,o &= 0,\\
 o\,n\,e &= 1,\\
 t\,w\,o &= 2,\\
 t\,h\,r\,e\,e &= 3,\\
 f\,o\,u\,r &= 4,\\
 f\,i\,v\,e &= 5,\\
 s\,i\,x &= 6,\\
 s\,e\,v\,e\,n &= 7,\\
 e\,i\,g\,h\,t &= 8,\\
 n\,i\,n\,e &= 9,\\
 t\,e\,n &= 10,\\
 e\,l\,e\,v\,e\,n &= 11,\\
 n\,e\,g\,a\,t\,i\,v\,e &= -1.
\end{align*}

\bigskip
\noindent\textbf{1. From the first equation:}
\[
 z\,e\,r\,o = 0
\]
\[
 \text{Since }e,r,o\neq 0,\text{ we conclude }
 z = 0.
\]

\bigskip
\noindent\textbf{2. Introduce free parameters:}
\[
 o,\;e,\;r,\;v
\]
be four arbitrary nonzero real numbers.

\bigskip
\noindent\textbf{3. From the equation}
\[
 o\,n\,e = 1
 \quad\Longrightarrow\quad
 n = \frac{1}{oe}.
\]

\bigskip
\noindent\textbf{4. From the equation}
\[
 t\,e\,n = 10
 \quad\Longrightarrow\quad
 t = 10\,o.
\]

\bigskip
\noindent\textbf{5. From the equation}
\[
 (10o)\,w\,o = 2
 \quad\Longrightarrow\quad
 w = \frac{1}{5\,o^2}.
\]

\bigskip
\noindent\textbf{6. From the equation}
\[
 (10o)\,h\,r\,e^2 = 3
 \quad\Longrightarrow\quad
 h = \frac{3}{10\,o\,r\,e^2}.
\]

\bigskip
\noindent\textbf{7. From the equation}
\[
 \Bigl(\frac{1}{oe}\Bigr)\,i\,\Bigl(\frac{1}{oe}\Bigr)\,e = 9
 \quad\Longrightarrow\quad
 i = 9\,o^2e.
\]

\bigskip
\noindent\textbf{8. From the equation}
\[
 f\,(9o^2e)\,v\,e = 5
 \quad\Longrightarrow\quad
 f = \frac{5}{9\,o^2\,e^2\,v}.
\]

\bigskip
\noindent\textbf{9. From the equation}
\[
 f\,o\,u\,r = 4
 \quad\Longrightarrow\quad
 u = \frac{4}{f\,o\,r}
 = \frac{36\,o\,e^2\,v}{5\,r}.
\]

\bigskip
\noindent\textbf{10. From the equation}
\[
 s\,e^2\,v\,n = 7
 \quad\Longrightarrow\quad
 s = \frac{7\,o}{e\,v}.
\]

\bigskip
\noindent\textbf{11. From the equation}
\[
 \Bigl(\frac{7o}{e\,v}\Bigr)\,(9o^2e)\,x = 6
 \quad\Longrightarrow\quad
 x = \frac{2\,v}{21\,o^3}.
\]

\bigskip
\noindent\textbf{12. From the equation}
\[
 e\,(9o^2e)\,g\,\Bigl(\frac{3}{10\,o\,r\,e^2}\Bigr)\,(10o) = 8
 \quad\Longrightarrow\quad
 g = \frac{8\,r}{27\,o^2}.
\]

\bigskip
\noindent\textbf{13. From the equation}
\[
 e^3\,l\,v\,n = 11
 \quad\Longrightarrow\quad
 l = \frac{11\,o}{e^2\,v}.
\]

\bigskip
\noindent\textbf{14. From the equation}
\[
 \Bigl(\frac{1}{oe}\Bigr)\,e^2\,\Bigl(\frac{8r}{27\,o^2}\Bigr)\,a\,(10o)\,(9o^2e)\,v = -1
 \quad\Longrightarrow\quad
 a = -\frac{3}{80\,e^2\,r\,v}.
\]

\bigskip
\section*{General Solution Summary}
\[
\begin{aligned}
 z &= 0, 
 & o,e,r,v &\text{ free parameters},\\
 n &= \frac{1}{oe}, 
 & t &= 10\,o, 
 & w &= \frac{1}{5\,o^2}, 
 & h &= \frac{3}{10\,o\,r\,e^2},\\
 i &= 9\,o^2e, 
 & f &= \frac{5}{9\,o^2\,e^2\,v}, 
 & u &= \frac{36\,o\,e^2\,v}{5\,r}, 
 & s &= \frac{7\,o}{e\,v},\\
 x &= \frac{2\,v}{21\,o^3}, 
 & g &= \frac{8\,r}{27\,o^2}, 
 & l &= \frac{11\,o}{e^2\,v}, 
 & a &= -\frac{3}{80\,e^2\,r\,v}.
\end{aligned}
\]
\section*{Concrete Example}
We choose
\[
 o = 1,\quad
 e = 1,\quad
 r = 1,\quad
 v = 1.
\]
Then
\[
\begin{aligned}
 z &= 0,\\
 o &= 1, & e &= 1, & r &= 1, & v &= 1,\\
 n &= 1, & t &= 10, & w &= \frac{1}{5}, & h &= \frac{3}{10},\\
 i &= 9, & f &= \frac{5}{9}, & u &= \frac{36}{5}, & s &= 7,\\
 x &= \frac{2}{21}, & g &= \frac{8}{27}, & l &= 11, & a &= -\frac{3}{80}.
\end{aligned}
\]

\section*{Why not Twelve?}

\newtheorem*{theorem}{Theorem}

\begin{theorem}
The system of word-product equations admits solutions in real numbers, but it is \emph{impossible} to extend it by
\[
t\,w\,e\,l\,v\,e = 12.
\]
\end{theorem}

\begin{proof}
From the equations
\[
o\,n\,e = 1
\quad\text{and}\quad
z\,e\,r\,o = 0,
\]
since \(o,n,e\neq0\) (else \(o\,n\,e\) could not be 1), we must have
\[
z = 0,
\quad
o = n = e = 1.
\]
Next, from
\[
t\,e\,n = 10
\quad\Longrightarrow\quad
t \cdot 1 \cdot 1 = 10
\quad\Longrightarrow\quad
t = 10,
\]
and from
\[
t\,w\,o = 2
\quad\Longrightarrow\quad
10 \,w \cdot 1 = 2
\quad\Longrightarrow\quad
w = \frac{1}{5}.
\]
Likewise, from
\[
e\,l\,e\,v\,e\,n = 11
\quad\Longrightarrow\quad
1 \cdot l \cdot 1 \cdot v \cdot 1 \cdot 1 = 11
\quad\Longrightarrow\quad
l\,v = 11.
\]
Hence the would-be “twelve” product is
\[
t \, w \, e \, l \, v \, e
= 
\bigl(t\bigr)\bigl(w\bigr)\bigl(e\bigr)\bigl(l\,v\bigr)\bigl(e\bigr)
=
10 \times \frac15 \times 1 \times 11 \times 1
=
2 \times 11
=
22.
\]
This shows that in any solution of the original system one necessarily has
\[
t\,w\,e\,l\,v\,e = 22 \neq 12.
\]
Therefore the equation \(t\,w\,e\,l\,v\,e = 12\) cannot be satisfied simultaneously
with the first eleven equations.
\end{proof}

\end{document}